\documentclass[tikz]{standalone}

\usepackage{neuralnetwork}

\newcommand{\xin}[2]{$x_#2$}
\newcommand{\xout}[2]{$\hat x_#2$}
\newcommand{\customout}[2]{
  \ifcase#2\relax
    % case 0, usually not used because index starts from 1
  \or $\hat{x}_1$  % For first neuron
  \or $\alpha$     % For second neuron
  \or $\beta$      % For third neuron
  \or $\gamma$     % For fourth neuron
  \or $\delta$     % For fifth neuron
  \or $\epsilon$   % For sixth neuron
  \or $\zeta$      % For seventh neuron
  \or $\eta$       % For eighth neuron
  \fi
}


\begin{document}
\begin{neuralnetwork}[height=8]
  \tikzstyle{input neuron}=[neuron, fill=orange!70];
  \tikzstyle{output neuron}=[neuron, fill=blue!60!black, text=white];
  \tikzstyle{every neuron}=[circle, minimum size=17pt, inner sep=0pt]
  \tikzstyle{annot} = [text width=4em, text centered]

  \inputlayer[count=8, bias=false, title=Input Layer, text=\xin]

  \hiddenlayer[count=5, bias=false, title=Jaden Mu's Brain]
  \linklayers

  \hiddenlayer[count=4, bias=false, title=Latent\\Representation]
  \linklayers

  \hiddenlayer[count=7, bias=false]
  \linklayers

  \outputlayer[count=8, title=Output Layer, text=\customout]
  \linklayers

\end{neuralnetwork}
\end{document}
