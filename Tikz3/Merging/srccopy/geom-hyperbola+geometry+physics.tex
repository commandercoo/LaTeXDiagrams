% https://wiki.physik.uzh.ch/cms/latex:tikz:hyperbola
    % Author: Izaak Neutelings (July, 2017)

\documentclass{article}
\usepackage{amsmath} % for \dfrac
\usepackage{tikz}
\tikzset{>=latex} % for LaTeX arrow head

\usepackage{pgfplots}                                     % for the axis environment

\usetikzlibrary{calc} % to do arithmetic with coordinates
\usetikzlibrary{angles,quotes} % for pic

% colors
\definecolor{mylightgrey}{RGB}{230,230,230}
\definecolor{mygrey}{RGB}{190,190,190}
\definecolor{mydarkgrey}{RGB}{110,110,110}
\definecolor{mygreen}{RGB}{120,220,160}
\definecolor{mydarkgreen}{RGB}{60,120,60}
\definecolor{myverydarkgreen}{RGB}{20,60,20}
\definecolor{mydarkred}{RGB}{140,40,40}

% mark right angle
\newcommand{\MarkRightAngle}[4][1.5mm]
{\coordinate (tempa) at ($(#3)!#1!(#2)$);
	\coordinate (tempb) at ($(#3)!#1!(#4)$);
	\coordinate (tempc) at ($(tempa)!0.5!(tempb)$);%midpoint
	\draw (tempa) -- ($(#3)!2!(tempc)$) -- (tempb);
}

% split figures into pages
\usepackage[active,tightpage]{preview}
\PreviewEnvironment{tikzpicture}
\setlength\PreviewBorder{1pt}%

\begin{document}
	
	
% RUTHERFORD SCATTERING - hyperbola orbit
\begin{tikzpicture}[scale=1]

	% limits & parameters
	\def\xa{-2.4}
	\def\xb{ 4}
	\def\ya{-4}
	\def\yb{ 4}
	\def\tmax{2.1}
	\def\a{1.3}
	\def\b{1}
	\def\c{{sqrt(\a^2+\b^2)}}
	\def\N{100} % number of points

	% coordinates
	\coordinate (O)  at (   0,  0 );
	\coordinate (A)  at (  \a,  0 );
	\coordinate (F1) at (  {sqrt(\a^2+\b^2)},   0 );
	\coordinate (F2) at ( -{sqrt(\a^2+\b^2)},   0 );
	\coordinate (P)  at ( -{\a^2/sqrt(\a^2+\b^2)}, -{\a*\b/sqrt(\a^2+\b^2)} );
	\coordinate (P1) at (\xb*\a, \yb*\b);
	\coordinate (P2) at (\xb*\a,-\yb*\b);
	\coordinate (yshift) at (0,0.4);

	% axes & asymptotes
	\draw[mygrey] % x axis
	(\xa*\a,0) -- (\xb*\a,0);
	\draw[dashed,mydarkgrey]
	(-\xb*\a*0.45, \ya*\b*0.45) -- (\xb*\a, \yb*\b)
	(-\xb*\a*0.45,-\ya*\b*0.45) -- (\xb*\a,-\yb*\b);

	% arrows
	\def\vtheta{30}
	\def\vradius{0.8}
	\draw[->,myverydarkgreen,shift=($(P1)-(yshift)$),scale=0.6]
	(0,0) -- (-\a,-\b) node[midway,below right=0pt] {${v}_i$};
	\draw[->,myverydarkgreen,shift=($(P2)+(yshift)$),scale=0.6]
	(-\a,\b) -- (0,0) node[midway,above right=-2pt] {${v}_f$};
	\draw[->,myverydarkgreen]
	(\a+0.35,{\vradius*sin(\vtheta)}) arc (180-\vtheta:180+\vtheta+10:\vradius)
	node[above right=0pt] {$v^*$};

	% angles
	\MarkRightAngle{F2}{P}{O}
	\pic [draw,myverydarkgreen,"$\theta$",angle radius=12,angle eccentricity=1.4] {angle = F2--O--P};
	\pic [draw,below,angle radius=16,angle eccentricity=1.4] {angle = P2--O--P1};
	\node[right=1pt,below=-2pt,myverydarkgreen] at ($(O)!0.5!(A)$)  {\small$\phi$};

	% hyperbola
	\draw[color=mylightgrey,line width=0.5,samples=\N,variable=\t,domain=-\tmax*0.58:\tmax*0.58] % left
	plot({-\a*cosh(\t)},{\b*sinh(\t)});
	\draw[color=mydarkgreen,line width=1,samples=\N,variable=\t,domain=-\tmax:\tmax] % right
	plot({ \a*cosh(\t)},{\b*sinh(\t)}); % {exp(\y)+exp(-\y)

	% nodes
	\draw[myverydarkgreen]
	(F2) -- (P) node[midway,below left=1pt,] {$b$};
	\fill[radius=1.5pt]
	(A)  circle node[above left=0pt] {A}
	(O)  circle node[above=2pt] {O};
	\fill[radius=2.5pt,mydarkred]
	(F2) circle node[above=2pt] {N};
	\draw[]
	%(O) -- (F2)
	node[left=1pt,above=0pt] at ($(F2)!0.5!(O)$) {$c$}
	node[below right] at ($(P)!0.5!(O)$)  {$a$};

	% alpha particle
	\draw[radius=1pt,mydarkgreen,fill]
	({ \a*cosh(\tmax*1.02)},{\b*sinh(\tmax*1.02)}) circle node[above right=0pt] {$\alpha$};

\end{tikzpicture}
	
	
% RUTHERFORD SCATTERING - hyperbolic orbits with different impact parameters
\begin{tikzpicture}[scale=1]

	% limits & parameters
	\def\xa{-35}
	\def\xb{ 55}
	\def\ya{ -1}
	\def\yb{ 55}
	\def\tmax{4.5}
	\def\N{50} % number of points

	% use axes to get square box which cuts of the long curves
	\begin{axis}[ xmin=\xa,xmax=\xb,
		ymin=\ya,ymax=\yb,
		hide x axis, hide y axis,
		xticklabels={,,},yticklabels={,,}
		axis line style={draw=none}, tick style={draw=none}
		]

		% loop over multiples \u of impact parameters \b=\u*0.25
		\def\a{1}
		\foreach \u in {1,3,6,10,15,21,28,38}{
			\def\b{\u*0.25}
			\def\c{sqrt(\a^2+\b^2)}
			% hyperbola
			\addplot[color=mydarkgreen,line width=0.5,samples=\N,smooth,variable=\t,domain=-\tmax:\tmax]
			({   \a/\c*(-\a*cosh(\t)-\c) + \b/\c*\b*sinh(\t)  },
			{  -\b/\c*(-\a*cosh(\t)-\c) + \a/\c*\b*sinh(\t)  });
		}

		% nucleus
		\addplot[mydarkred,mark=*,mark size=2pt,mark options=solid] coordinates {(0,0)};

	\end{axis}

\end{tikzpicture}
	
	
	
	
	
\end{document}

