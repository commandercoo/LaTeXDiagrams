\documentclass{standalone}
\usepackage{tikz}
\begin{document}
\pagestyle{empty}

\def\layersep{3cm}
\def\nodeinlayersep{1.5cm}

\begin{tikzpicture}
  [
    shorten >=1pt,->,
    draw=black!50,
    node distance=\layersep,
    every pin edge/.style={<-,shorten <=1pt},
    neuron/.style={circle,fill=black!25,minimum size=17pt,inner sep=0pt},
    input neuron/.style={neuron, fill=green!50,},
    output neuron/.style={neuron, fill=red!50},
    hidden neuron/.style={neuron, fill=blue!50},
    annot/.style={text width=4em, text centered},
    bias/.style={neuron, fill=yellow!50,minimum size=4em},%<-- added %%%
  ]

  % Draw the input layer nodes
  \foreach \name / \y in {1,...,3}
    \node[input neuron, pin=left:Input \#\y] (I-\name) at (0,-\y-2.5) {};  

  % set number of hidden layers
  \newcommand\Nhidden{2}

  % Draw the hidden layer nodes
  \foreach \N in {0,...,\Nhidden} {
      \foreach \y in {0,...,5} { % <-- added 0 instead of 1 %%%%%
      \ifnum \y=4
      \ifnum \N>0 %<-- added %%%%%%%%%%%%%%%%%%%%%%%%%%%%%%%%%%%%%%%%%%%%
        \node at (\N*\layersep,-\y*\nodeinlayersep) {$\vdots$};  % add dots
        \else\fi %<-- added %%%%%%%%%%%%%%%%%%%%%%%%%%%%%%%%%%%%%%%%%%%%
      \else
          \ifnum \y=0 %<-- added %%%%%%%%%%%%%%%%%%%%%%%%%%%%%%%%%%
          \ifnum \N<3 %<-- added %%%%%%%%%%%%%%%%%%%%%%%%%%%%%%%%%%
            \node[bias] (H\N-\y) at (\N*\layersep,-\y*\nodeinlayersep ) {Bias}; %<-- added
            \else\fi %<-- added %%%%%%%%%%%%%%%%%%%%%%%%%%%%%%%%
          \else %<-- added %%%%%%%%%%%%%%%%%%%%%%%%%%%%%%%%%%%%%%%%%%%%
            \ifnum \N>0 %<-- added %%%%%%%%%%%%%%%%%%%%%%%%%
            % print function
            \node[hidden neuron] (H\N-\y) at (\N*\layersep,-\y*\nodeinlayersep ) {$\frac{1}{1+e^{-x}}$}; %<-- added %%%%%%%%%%%
                \else\fi %<-- added %%%%%%%%%%%%
          \fi %<-- added %%%%%%%
      \fi
    }
      \ifnum \N>0 %<-- added %%%%%%
      % print hidden layer labels at the top
    \node[annot,above of=H\N-1, node distance=1cm,yshift=2cm] (hl\N) {Hidden layer \N}; % <- added yshift=2cm %%%%%%%%%%%%
    \else\fi %<-- added %%%%%
  }

  % Draw the output layer node and label
  \node[output neuron,pin={[pin edge={->}]right:Output}, right of=H\Nhidden-3] (O) {}; 
  
  % Connect bias every node in the input layer with every node in the
  % hidden layer.
  \foreach \source in {1,...,3}
      \foreach \dest in {1,...,3,5} {
        % \path[yellow] (H-0) edge (H1-\dest);
        \path[dashed,orange] (H0-0) edge (H1-\dest); %<-- added %%%%%
          \path[green!50] (I-\source) edge (H1-\dest);  % change to green, yellow gets blended
    };

  % connect all hidden stuff
  \foreach [remember=\N as \lastN (initially 1)] \N in {2,...,\Nhidden}
      \foreach \source in {0,...,3,5} 
          \foreach \dest in {1,...,3,5}{
              \ifnum \source=0 %<-- added %%%%%%%%%%%%%%%%%%%%%%%
          \path[dashed,red](H\lastN-\source) edge (H\N-\dest);%<-- added 
            \else %<-- added %%%
            \path[blue!50] (H\lastN-\source) edge (H\N-\dest);%<-- added 
            \fi %<-- added %%%
            }; %<-- added %%%%


  % Connect every node in the hidden layer with the output layer
  \foreach \source in {1,...,3,5}
    \path[green!50] (H\Nhidden-\source) edge (O);
    \path[dashed,red] (H2-0) edge (O); %<-- added %%%%

% Annotate the input and output layers
  \node[annot,left of=hl1] {Input layer};
  \node[annot,right of=hl\Nhidden] {Output layer};  
\end{tikzpicture}
% End of code
\end{document}