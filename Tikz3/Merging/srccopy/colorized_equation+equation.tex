% https://www.overleaf.com/read/cvmtqywqgvvw#/43203532/
% https://betterexplained.com/articles/colorized-math-equations/

% \documentclass{article}
\documentclass[preview]{standalone}
\usepackage[utf8]{inputenc}
\usepackage{amsmath}
\usepackage{amssymb}
\usepackage{color}

\newcommand{\plain}{\color{black}}
\newcommand{\Frac}[2]{\genfrac{}{}{1pt}{}{#1}{#2}}	% thicker fraction line

\definecolor{c1}{RGB}{114,0,172}   % primary
\definecolor{c2}{RGB}{45,177,93}   % true
\definecolor{c3}{RGB}{251,0,29}    % false
\definecolor{c4}{RGB}{18,110,213}  % secondary
\definecolor{c5}{RGB}{255,160,109} % tertiary
\definecolor{c6}{RGB}{219,78,158}  % alt-primary 

\renewcommand{\familydefault}{\sfdefault}

\begin{document}
\begin{center}

\newcommand{\growth}{\color{c1}}
\newcommand{\unitQuantity}{\color{c2}}
\newcommand{\unitInterest}{\color{c3}}
\newcommand{\unitTime}{\color{c4}}
\newcommand{\perfectly}{\color{c5}}
\newcommand{\compounded}{\color{c6}}

$$\growth e
\plain =
\perfectly \lim_{n\to\infty}
\plain \left(
\unitQuantity 1 + \unitInterest \frac{1}{\compounded n}
\plain \right)
\unitTime^{1 \cdot \compounded n}
$$

\growth       The base for continuous growth
\plain        is
\\
\unitQuantity the unit quantity 
\unitInterest earning unit interest
\unitTime     for unit time, 
\\
\compounded   compounded
\perfectly    as fast as possible

\end{center}
\end{document}
