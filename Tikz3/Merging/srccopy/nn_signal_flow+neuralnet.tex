% https://newbiettn.github.io/2016/12/16/tikz/
% Unit circle
% Author: The Author
% What this does
\documentclass[landscape]{article}
\usepackage{tikz}

% DON'T FORGET TO ADD THOSE IN THE HEADER OF LATEX FILES
%\usepackage{tikz}
\usetikzlibrary{arrows,backgrounds}
\usepgflibrary{shapes.multipart}

%%%<
\usepackage{verbatim}
\usepackage[active,tightpage]{preview}
\PreviewEnvironment{tikzpicture}
\setlength\PreviewBorder{5pt}%
%%%>

\begin{comment}
	:Title: the title
	
	Describe the application
\end{comment}

\usepackage[top=1in,bottom=1in,right=1in,left=1in]{geometry}

\begin{document}


\begin{tikzpicture}
\tikzstyle{rectangle_style}=[rectangle, draw]
\tikzstyle{dividedrectangle_style}=[draw, rectangle split, rectangle split parts=2, rotate = 90, minimum height = 15mm, minimum width = 10mm]

% neuron i
\foreach \x in {0,...,2}
\draw node at (0, -\x) [rectangle_style] (neuron_i_\x) {$y_\x$};
\foreach \x in {1,...,3}
\fill (0, -2.5 - \x*0.15) circle (1pt);
\draw node at (0, -3.5) [rectangle_style] (neuron_i_3) {$y_i$};

% w_ji
\foreach \x in {0,...,2}
\draw node at (1.5, -\x) [] (w_ji_\x) {$w_{j\x}$};
\draw node at (1.5, -3.5) [] (w_ji_i) {$w_{ji}$};
\foreach \x in {1,...,3}
\fill (1.5, -2.5 - \x*0.15) circle (1pt);

% neuron j
\node at (6.5, -1.5) [dividedrectangle_style] (neuron_j){\rotatebox{-90}{$s_j = \sum {w_{ji}y_i}$} \nodepart{second} \rotatebox{-90}{$y_j = \varphi (s_j)$}};
\foreach \x in {1,...,3}
\fill (6.5, -2.25 - \x*0.15) circle (1pt);
\foreach \x in {1,...,3}
\fill (6.5,  - 0.15 - \x*0.15) circle (1pt);

% output
\node at (10, -1.5) [circle, draw, fill=black] (output) {$$};

% desired output
\node at (10, 0) [] (desired_output) {$d_j$};

% error
\node at (12, -1.5) [] (error) {$e_j = d_j - y_j$};

% connect: y_i -> w_ji
\foreach \i in {0,...,2}
\path[-] (neuron_i_\i) edge node[] {} (w_ji_\i);
\path[-] (neuron_i_3) edge node[] {} (w_ji_i);

% connect: w_ji -> neuron j
\foreach \i in {0,...,2}
\path[->] (w_ji_\i) edge node[] {} (neuron_j);
\path[->] (w_ji_i) edge node[] {} (neuron_j);

% connect: neuron j -> output
\path[->] (neuron_j) edge node[above, midway] {$$} (output);

% connect: desired output -> output
\path[->] (desired_output) edge node[] {} (output);

% connect: output -> error
\path[->] (output) edge node[] {} (error);

\end{tikzpicture}
\end{document}
