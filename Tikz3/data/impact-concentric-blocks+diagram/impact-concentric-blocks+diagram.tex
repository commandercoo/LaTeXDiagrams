% https://github.com/FriendlyUser/LatexDiagrams
\documentclass[border=5pt]{standalone}
\usepackage{xcolor}

	\definecolor{ocre}{HTML}{800000}
	\definecolor{sky}{HTML}{C6D9F1}
	\definecolor{skybox}{HTML}{5F86B3}

\usepackage{tikz}
\usepackage{pgfmath}
\usetikzlibrary{decorations.text, arrows.meta,calc,shadows.blur,shadings}

\renewcommand*\familydefault{\sfdefault} % Set font to serif family

% arctext from Andrew code with modifications:
%Variables: 1: ID, 2:Style 3:box height 4: Radious 5:start-angl 6:end-angl 7:text {format along path} 
\def\arctext[#1][#2][#3](#4)(#5)(#6)#7{

\draw[#2] (#5:#4cm+#3) coordinate (above #1) arc (#5:#6:#4cm+#3)
             -- (#6:#4) coordinate (right #1) -- (#6:#4cm-#3) coordinate (below right #1) arc (#6:#5:#4cm-#3) coordinate (below #1)
             -- (#5:#4) coordinate (left #1) -- cycle;
            \def\a#1{#4cm+#3}
            \def\b#1{#4cm-#3}
\path[
    decoration={
        raise = -0.5ex, % Controls relavite text height position.
        text  along path,
        text = {#7},
        text align = center,        
    },
    decorate
    ]
    (#5:#4) arc (#5:#6:#4);
}

%arcarrow, this is mine, for beerware purpose...
%Function: Draw an arrow from arctex coordinate specific nodes to another 
%Arrow start at the start of arctext box and could be shifted to change the position
%to avoid go over another box.
%Var: 1:Start coordinate 2:End coordinate 3:angle to shift from acrtext box  
\def\arcarrow(#1)(#2)[#3]{
    \draw[thick,->,>=latex] 
        let \p1 = (#1), \p2 = (#2), % To access cartesian coordinates x, and y.
            \n1 = {veclen(\x1,\y1)}, % Distance from the origin
            \n2 = {veclen(\x2,\y2)}, % Distance from the origin
            \n3 = {atan2(\y1,\x1)} % Angle where acrtext starts.
        in (\n3-#3: \n1) -- (\n3-#3: \n2); % Draw the arrow.
}

\begin{document}
\begin{tikzpicture}[
    % Environment Cfg
    font=\sf    \scriptsize,
    % Styles
    myarrow/.style={
        thick,
        -latex,
    },
    Center/.style ={
        circle,
        fill=ocre,
        text=white,
        align=center,
        font =\footnotesize\bf,
        inner sep=1pt,          
    },
    RedArc/.style ={
        color=black,
        thick,
        fill=ocre,
        blur shadow, %Tikzedt not suport online view
    },
    SkyArc/.style ={
        color=skybox,
        thick,
        fill=sky,
        blur shadow, %Tikzedt not suport online view
    },
    ]

    % Drawing the center
    \node[Center](SOSA) at (0,0) { Sensor \\ Observation, \\ Sample, and \\ Actuator \\(SOSA)};
    \coordinate (SOSA-R) at (0:1.2); % To make compatible with \arcarrow macro.

    % Drawing the Tex Arcs

    % \Arctext[ID][box-style][box-height](radious)(start-angl)(end-angl){|text-styles| Text}

    \arctext[SSN][RedArc][8pt](2.25)(180)(60){|\footnotesize\bf\color{white}| Semantic Sensor Network (SSN)};
    \arctext[SCap][RedArc][8pt](2.25)(50)(-20){|\footnotesize\bf\color{white}| System Capabilities};
    \arctext[SRel][SkyArc][8pt](2.25)(190)(255){|\footnotesize\color{black}| System Relation};
    \arctext[OMAM][RedArc][5pt](3.5)(205)(265){|\scriptsize\bf\color{white}| O{\&}M Alignment Module};
    \arctext[PROV][SkyArc][5pt](3.5)(270)(320){|\scriptsize| PROV Alignment Module};
    \arctext[OBOE][SkyArc][5pt](3.5)(-35)(20){|\scriptsize| OBOE Alignment Module};
    \arctext[DUAM][SkyArc][5pt](4.5)(215)(150){|\scriptsize| Dolce-UltraLite Alingment Module};
    \arctext[SSNX][SkyArc][5pt](4.5)(145)(80){|\scriptsize| SSNX Alingment Module};

    %ADITIONAL
    \arctext[NEW][
        color=white,
        shade,      
        upper left=red,
        upper right=black!50,
        lower left=blue,
        lower right=blue!50,
        rounded corners = 8pt
        ][8pt](5.2)(100)(-20){|\footnotesize\bf\color{white}| You can create and use all the style options for shapes and text};

    %Drawing the Arrows
    %\arcarrow(above/below ID)(abobe/below ID)[shift]
    \arcarrow(below DUAM)(above SRel)[15];
    \arcarrow(below SSNX)(above SSN)[35];
    \arcarrow(below SSN)(SOSA-R)[60];
    \arcarrow(below right OMAM)(SOSA-R)[4];
    \arcarrow(below right PROV)(SOSA-R)[25];
    \arcarrow(below OBOE)(SOSA-R)[-5];

    %Same level Arrows
    \draw[myarrow] (left SSNX) -- (right DUAM);
    \draw[myarrow] (left SSN) -- (left SRel);
    \draw[myarrow] (left SCap) -- (right SSN);

        \draw[myarrow] (-5,-3.5) coordinate (legend) -- ++(.8,0) node[anchor=west] {owl: imports (extends)};
        \draw[RedArc] (legend)++(0,-0.4) rectangle ++(.8,-.3)++(0,.2) node[anchor=west] {normative};
        \draw[SkyArc] (legend)++(0,-1) rectangle ++(.8,-.3)++(0,.2) node[anchor=west, color=black] {non-normative};

\end{tikzpicture}

\end{document}