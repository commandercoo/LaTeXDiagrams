% http://pgfplots.net/media/tikz/examples/TEX/plot-markers.tex
\documentclass[border=10pt]{standalone}
%%%<
\usepackage{verbatim}
\usepackage{pgfplots}
\pgfplotsset{width=7cm, compat=1.8, grid style={dashed}}

\begin{comment}
:Title: Markers in a line diagram
:Tags: 2D;Markers
:Author: Elke Schubert
:Slug: plot-markers

Besides x and y values, special markers illustrate the result.

This topic was discussed on: http://texwelt.de/wissen/fragen/3363/

Data file
-------
if you want the data to be saved as you modify this file, add this above the comments:

	\usepackage{filecontents}
	
	% changes in this table make changes to CSV file
	\begin{filecontents}{data.csv}
	measuring   power1   result1   power2   result2
	1           2        0.2       3.5      0.39
	2           3        0.35      3.8      0.3
	3           4        0.45      7.9      0.35
	4           5        0.5       8.5      0.39
	5           6        0.65      8.0      0.38
	6           7        0.68      8.5      0.4
	7           8        0.7       9.5      0.41
	8           7.5      0.73      10.5     0.99
	9           6.7      0.75      {}       {}
	10          10       0.79      {}       {}
	\end{filecontents}
	%%%>

\end{comment}

\begin{document}
\begin{tikzpicture}
  \begin{axis}
    [
      scatter,
      scatter src = explicite,
      grid        = major, % draws coordinate grid
      xlabel      = Force $\lbrack{}$ F $\rbrack$, 
      ylabel      = Amperage $\lbrack{}$ kA $\rbrack$,
      width       = \linewidth,
      height      = 10cm,
      xmin = 0, xmax = 15,
      ymin = 0, ymax = 12,
    ]
    \addplot+ [
      visualization depends on =
        {10*\thisrow{result1} \as \perpointmarksize},
      scatter/@pre marker code/.append style =
        {/tikz/mark size = \perpointmarksize},
    ]
    % read the table
    % read the column result1
    table[x=measuring, y=power1, point meta=\thisrow{result1}] {./data/data.csv};
    \addplot+ [
      visualization depends on =
        {10*\thisrow{result2} \as \perpointmarksize},
      scatter/@pre marker code/.append style =
        {/tikz/mark size = \perpointmarksize}
    ]
    % read the column result2
    table [x=measuring, y=power2, point meta=\thisrow{result2}] {./data/data.csv};
    \fill [gray, fill opacity=0.25] (axis cs:5,0) rectangle (axis cs:7,12);
  \end{axis}
\end{tikzpicture}
\end{document}