% Branches of Artificial Intelligence
% Drawn by Alfonso R. Reyes
% November 27, 2020
% Personal interpretation of "What is AI? / Branches of AI" article by Prof. John McCarthy.
% Original diagram: https://github.com/FriendlyUser/LatexDiagrams
\documentclass[border=5pt]{standalone}
\usepackage{xcolor}
\usepackage{ulem}


\definecolor{ocre}{HTML}{800000}
\definecolor{sky}{HTML}{C6D9F1}
\definecolor{skybox}{HTML}{5F86B3}

\usepackage{tikz}
\usepackage{pgfmath}
\usetikzlibrary{decorations.text, arrows.meta,calc,shadows.blur,shadings}

\renewcommand*\familydefault{\sfdefault} % Set font to serif family

% arctext from Andrew code with modifications:
%Variables: 1: ID, 2:Style 3:box height 4: Radious 5:start-angl 6:end-angl 7:text {format along path} 
\def\arctext[#1][#2][#3](#4)(#5)(#6)#7{
	\draw[#2] (#5:#4cm+#3) coordinate (above #1) arc (#5:#6:#4cm+#3)
	             -- (#6:#4) coordinate (right #1) -- (#6:#4cm-#3) coordinate (below right #1) arc (#6:#5:#4cm-#3) coordinate (below #1)
	             -- (#5:#4) coordinate (left #1) -- cycle;
	            \def\a#1{#4cm+#3}
	            \def\b#1{#4cm-#3}
	\path[
	    decoration={
	        raise = -0.5ex, % Controls relavite text height position.
	        text  along path,
	        text = {#7},
	        text align = center,        
	    },
	    decorate
	    ]
	    (#5:#4) arc (#5:#6:#4);
}

% arcarrow, this is mine, for beerware purpose...
% Function: Draw an arrow from arctex coordinate specific nodes to another 
% Arrow start at the start of arctext box and could be shifted to change the position
% to avoid go over another box.
% Var: 1:Start coordinate 2:End coordinate 3:angle to shift from acrtext box  
\def\arcarrow(#1)(#2)[#3]{
    \draw[very thick,->,>=latex,black!60] 
        let \p1 = (#1), \p2 = (#2), % To access cartesian coordinates x, and y.
            \n1 = {veclen(\x1,\y1)}, % Distance from the origin
            \n2 = {veclen(\x2,\y2)}, % Distance from the origin
            \n3 = {atan2(\y1,\x1)} % Angle where acrtext starts.
        in (\n3-#3: \n1) -- (\n3-#3: \n2); % Draw the arrow.
}


\begin{document}
\begin{tikzpicture}[
	    % Environment Cfg
	    font=\sf    \scriptsize,
	    % Styles
	    myarrow/.style={
	        thick,
	        -latex,
            black!60,
	    },
	    Center/.style ={
	        circle,
	        fill=ocre,
	        text=white,
	        align=center,
	        font =\footnotesize\bf,
	        inner sep=1pt,          
	    },
	    RedArc/.style ={
	        color=black,
	        thick,
	        fill=ocre,
	        blur shadow, 
	    },
	    SkyArc/.style ={
	        color=skybox,
	        thick,
	        fill=sky,
	        blur shadow, 
	    },
    ]

    % Drawing the center for AI
    \node[Center](AI) at (0,0) { Artificial \\ Intelligence \\(AI)\\ Agent};
    \coordinate (AROUND) at (0:1.2); 

    % Drawing the Text Arcs
    % Format: \Arctext[ID][box-style][box-height](radious)(start-angl)(end-angl){|text-styles| Text}
   
	% Logic
    \arctext[LOGI][RedArc][8pt](2.5)(120)(70){|\footnotesize\bf\color{white}| Logic};
	\arcarrow(below LOGI)(AROUND)[25];
    
    % Search
    \arctext[SRCH][RedArc][8pt](2.5)(60)(20){|\footnotesize\bf\color{white}| Search};
	\arcarrow(below SRCH)(AROUND)[20];
	   
	% Pattern Recognition
	\arctext[RECO][RedArc][8pt](2.5)(-60)(10){|\footnotesize\bf\color{white}| Pattern Recognition};
	\arcarrow(below RECO)(AROUND)[-35];

    % Representation
    \arctext[REPR][RedArc][8pt](2.5)(240)(290){|\footnotesize\bf\color{white}| Representation};
	\arcarrow(below REPR)(AROUND)[-25];
    
    % Inference
    \arctext[INFE][RedArc][8pt](2.5)(190)(230){|\footnotesize\bf\color{white} | Inference};    
    \arcarrow(below INFE)(AROUND)[-20];
    
    % Reasoning
    \arctext[REAS][RedArc][8pt](2.5)(180)(130){|\footnotesize\bf\color{white} | Reasoning};
	\arcarrow(below REAS)(AROUND)[25];
        
	% Epistemology
    \arctext[EPIS][RedArc][8pt](3.30)(220)(260){|\footnotesize\bf\color{white}|  Epistemology};

	% Planning
    \arctext[PLAN][RedArc][8pt](3.30)(210)(160){|\footnotesize\bf\color{white}| Planning};  
    
    % Learning
    \arctext[LEAR][RedArc][8pt](3.30)(150)(110){|\footnotesize\bf\color{white}| Learning};  

	% Ontology
    \arctext[ONTO][RedArc][8pt](3.30)(90)(50){|\footnotesize\bf\color{white}| Ontology};  
    
    % Heuristics
    \arctext[HEUR][RedArc][8pt](3.30)(40)(-10){|\footnotesize\bf\color{white}| Heuristics};  

	% Genetic Programming
	\arctext[GENE][RedArc][8pt](3.30)(270)(330){|\footnotesize\bf\color{white}| Genetic Programming};  
	

%    %ADITIONAL EXTERNAL ARC
%   \arctext[NEW][
%       color=white,
%       shade,      
%       upper left=gray!7,
%       upper right=gray!7,
%       lower left=gray!7,
%       lower right=gray!7,
%       rounded corners = 8pt
%       ][8pt](5.2)(180)(0){|\footnotesize\bf\color{black}| Interpration of "What is AI? / Branches of AI" article by Prof. John McCarthy.};

    % Drawing the Arrows from contributing branch to AI
    % Format: \arcarrow(above/below ID)(abobe/below ID)[shift]
    \arcarrow(below LEAR)(AROUND)[25];
    \arcarrow(below ONTO)(AROUND)[25];
    \arcarrow(below HEUR)(AROUND)[25];
    \arcarrow(below GENE)(AROUND)[-25];
    \arcarrow(below EPIS)(AROUND)[-15];
    \arcarrow(below PLAN)(AROUND)[25];
       

	% Legend and labels
    \draw[myarrow] (-5,-5) coordinate (legend) -- ++(.8,0) node[anchor=west] {(contribution)};
    \draw[RedArc] (legend)++(0,-0.4) rectangle ++(.8,-.3)++(0,.2) node[anchor=west, text width=3em] {capabilities,\\disciplines};
    \draw[SkyArc] (legend)++(0,-1) rectangle ++(.8,-.3)++(0,.2) node[anchor=west, color=black] {subfield};
    % source, book and authors
    \node [text width=5.5cm] at (1,-5.6) {Source: \textit {"What is AI? / Branches of AI"} article by Prof. John McCarthy. Stanford University.};
    % copyright
	\node[text width=3cm] at (3.25,-6.25) {\begin{tiny}Copyright \textcopyright Alfonso R. Reyes, 2020\end{tiny}};

\end{tikzpicture}  
\end{document}
